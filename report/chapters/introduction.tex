\documentclass[../thesis.tex]{subfiles}
\begin{document}
\renewcommand*\thesection{\arabic{section}}

\section{Background}
The rapid development of information technology has led to its widespread use in various fields, including business operations. As such, the integration of software applications into sales business operations has become a vital need to improve management quality and increase revenue. However, choosing a suitable software application on a suitable platform (web, desktop application, mobile application) for business operations remains a difficult issue, particularly for small and medium-sized businesses.

\section{Problem statement}
Several sales management applications are available in the market that is suitable for typical business owners, such as KiotViet, a sales management software. Some medium-sized stores have also invested in a suitable management model through a website or computer application, along with supporting devices. \\ Nevertheless, most applications are commercial, which small store owners cannot afford to invest in, along with the necessary equipment. Therefore, they are forced to rely on manual bookkeeping methods to manage their sales.
\section{Aims and objectives}
% Mục tiêu chính của đề tài là tạo ra phầm mềm ứng dụng thu chi trên nền tảng di động, có tính năng quét mã vạch của sản phẩm. Cụ thể nhiệm vụ của đề tài là xây dựng ứng dụng di  động với thiết các chức năng chính như 
% Thực hiện tạo lập và theo dỗi hóa đơn
% Quản lý kho hàng sản phẩm của người dùng
% Quản lý thông tin sổ nợ
% Thông kê thu chi theo khoảng thời gian
The purpose of this project is to develop a software application that integrates with sales business operations to enhance management quality and increase revenue. This project aims to provide a cost-effective solution for small and medium-sized businesses to manage their sales operations. The developed software application will have a user-friendly interface that allows for easy integration and adoption by businesses.

\section{Research objects and scope}

\subsection{Research Objectives:}
% Đối tượng nghiên cứu
% Ứng dụng triển khai phần mềm ứng dụng trên nền tảng di động trên thư viện Flutter
% Nền tảng dữ liệu trên Firebase
% 
% \begin{enumerate}
%     \item Investigate challenges faced by small businesses in managing sales and inventory.
%     \item Develop a mobile application solution for small businesses.
%     \item Evaluate usability and effectiveness of the developed application.
%     \item Identify potential areas for improvement in the developed application.
%     \item Provide recommendations for further development and implementation of the application.

% \end{enumerate}
The research aims to develop and evaluate a mobile application solution for small businesses to assist in managing sales and inventory. The application will be developed using the Flutter framework and will utilize Firebase as the data platform. The specific objectives of the research are to:
\begin{itemize}
    \item[-] Investigate challenges faced by small businesses in managing sales and inventory.
    \item[-] Develop a mobile application solution for small businesses using the Flutter framework and Firebase.
    \item[-] Evaluate the usability and effectiveness of the developed application through user testing and surveys.    
\end{itemize}
The outcomes of the study will contribute to the knowledge of mobile application solutions for small businesses.

\subsection{Research Scope:}
% Các quy trình nghiệp vụ bán lẻ  của các chủ và cở sở kinh doanh tạp hóa nhỏ lẻ.
% Xây dựng  phần mềm ứng dụng di động đa nền tảng dựa trên thư viện Flutter và công nghệ Firebase của Google.
% This research will develop a mobile application solution to assist small businesses in managing sales and inventory, using a mixed-methods approach that includes surveys with small business owners. 
% The application will be developed using the Flutter framework and evaluated through user testing and surveys. 
% The study will focus on small businesses and will not consider larger organizations. 
% Recommendations for further development and implementation of the application will be provided. The outcomes of the study will contribute to the knowledge of mobile application solutions for small businesses.
\begin{itemize}
    \item[-] The research investigates the retail business processes of small grocery store owners and businesses. The scope of the study is limited to small businesses and will not consider larger organizations.
    \item[-] A cross-platform mobile application will be developed using the Flutter library and Google's Firebase technology.
\end{itemize}
%Solution approach

\section{Solution approach}
% Xây dựng phần mềm ứng dụng di động đa nền tảng Flutter
% Nghiên cứu các công nghệ của Firebase bao gồm  Firebase Authentication, Firebase Firestore và Firebase Storge.
% Các quy trình nghiệp vụ liên quan đến hoạt động bán lẻ của các chủ cửa hàng tạp hóa nhỏ lẽ.
% \textbf{Needs assessment}: Conduct a thorough analysis of the current sales management practices used by small businesses, identifying the challenges they face and the features they require in an effective sales management application.\\
% \textbf{Application design:} Based on the results of the needs assessment, design a mobile application using Flutter that integrates the required features and addresses the identified challenges.\\
% \textbf{Application development:} Develop the mobile application using Flutter, ensuring that it meets the identified requirements and is user-friendly for small business owners.
\begin{itemize}
    \item[-] Building a cross-platform mobile application using Flutter.
    \item[-] Researching Firebase technologies including Firebase Authentication, Firebase Firestore and Firebase Storage.
    \item[-] Business processes related to the retail operations of small grocery store owners.
\end{itemize}
%Summary
\section{Summary of contributions and achievements}
% Đề tài này cung cấp một giải pháp để quản lý mảng kinh doanh bán lẻ
% Xây dựng phần mêm ứng dụng di động đa nền tảng với giao diện thân thiện dễ dùng.
\begin{itemize}
    \item[-] This research provides a solution for managing the retail sales aspect of small businesses through the development of a cross-platform mobile application with a user-friendly interface. 
    \item[-] The application is designed to assist small businesses in managing sales and inventory, addressing the challenges they face in this area. 
\end{itemize}
%Organization
\section{Organization of the report}
% Nội dung của niên luận cơ sở bao gồm những phần sau:
% Phần giới thiệu: Trình bày về các vấn đề nảy sinh cần giải quyết, lịch sử giải quyết vấn đề, mục tiêu đề tài, đống góp của đề tài và cũng như nôi dung trong đề tài sẽ thực hiện.
% Phần nội dung:  Mô tả chi tiết bài toán, phân tích, đặc tả chức năng, cài đặt và thiết kế dữ liệu, giao diện cho ứng dụng và đánh giá  kiểm thử phần mềm.
%  Phần nội dung bao gồm các chương:
% Mô tả bài toán: Mô tả chi tiết bài toán, các chức năng, yêu cầu của đã đặt ra và cơ sở lý thuyết.
% Thiết kế và cài đặt giải pháp: Tổng quan hệ thống, thiết kế kiến trúc, thiết kế dữ liệu, thiết kế theo chức năng, và các sơ đồ giúp xây dựng hệ thống.
% Kiểm thử và đánh giá: Trình bày kế hoặc kiểm thử và quản lý kiểm thử và cá trường hợp kiểm thử các chức năng chính của hệ thống.
%Phần kết luận:  Trình bài những kết quả đạt được, các hạn chế còn tồn động và hướng phát triển của hệ thống.
This report includes the following sections:
{
\renewcommand\labelitemi{}
\begin{itemize}
    \item \textbf{Introduction:} This section provides an overview of the thesis, including the issues that need to be addressed, the history of problem-solving, the objectives of the thesis, the contribution of the thesis, and the content that will be covered.
    \item \textbf{Content:} This section includes a detailed description of the problem, analysis, functional specification, data design and implementation, interface for the application, and evaluation of software testing. The content section is divided into three chapters.
          
    \begin{itemize}
              \item[] \textbf{Chapter 1: Problem description.} This chapter provides a detailed description of the problem, including functions, requirements, and theoretical foundation.
              \item[] \textbf{Chapter 2: System design.} This chapter provides an overview of the system, including the architectural design, data design, functional design, and diagrams to help build the system.
              \item[] \textbf{Chapter 3: Implementation.} This chapter provides an implementation of the system design.
              \item[] \textbf{Chapter 4: Testing and evaluation.} This chapter presents the testing plan and management, testing scenarios for the main functions of the system.
          \end{itemize}
    \item \textbf{Conclusion:}  This section presents the results achieved, the remaining limitations, and the system's further development.
\end{itemize}
}
\end{document}
